\chapter{Conclusion}

The aim of this work was to identify disconnected data endpoints and figure out a way to connect them and make use of them. I demonstrated the need by highlighting the misunderstandings between different teams and pointed out where value could be more effectively driven.

In the second part, I analyzed the existing tools on the market and I assessed their usability in the given context. Not being fully satisfied with their functions, I took the initiative and designed a system that would be in-house, deployed in the AWS VPC environment. 

In the third and fourth part, I designed and developed the system into a fully viable product (= MVP). As I went along, I kept validating my ideas with my peers and supervisors.

In the last part, I set myself a goal to test the system from many aspects - idea, usability and performance. While some tests went better than others, it brought me so many valuable insights and highlighted aspects where there was room for improvement that I hadn't even considered before. For that I see the tests as an immense success, regardless of the actual results - they show that the idea is valid, people are interested in it and there is something to continue working on.

\section{Technical debt}

Naturally, when a project delivery is time-boxed, certain functions/features are given higher priority than others along the way. This project has been no exception.

\subsection{Semantic Data Manager}

\begin{enumerate}
	\item The SSL certificate is only self-signed.
	\item JIRA communication could be abstracted out into another layer to make sure when the REST API changes that the rewrites would only happen in the parser.
	\item There is no mechanism for versioning of the configurations.
\end{enumerate}

\subsection{Administration Application}

\begin{enumerate}
	\item Several UI elements need more fine-grained graphics to improve the usability.
	\item There should be a search bar in the issues.
	\item Dropbox integration is only one out of many other possibilities - Syncplicity, E-mail etc. should be supported as well.
\end{enumerate}

\subsection{Tracking Engine}

\begin{enumerate}
	\item The SSL certificate is only self-signed.
	\item There is no authentication method for the applications sending data.
	\item The analysis of the data development is currently limited to time series.
\end{enumerate}

\subsection{Timeseries Application}

\begin{enumerate}
	\item There is no user management built into the application.
	\item KPIs can't be added per KPI, only per application.
	\item There is no graph mash-up, only number analysis per day.
\end{enumerate}

\section{Future development}

There is absolutely no shortage of work that can be done on the project. The project is healthy and stable and there are plenty of features that can be implemented in order to drive this project even further. I am very optimistic thanks to the support of all the people that took interest in the project. 

One step at a time, the best thing to do is to take a look and revisit the foundations of the project: the Semantic Data Manager and the Tracking Engine. When these two components are free from technical debt and robust, only the sky is the limit.