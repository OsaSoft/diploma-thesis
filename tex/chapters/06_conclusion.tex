\chapter{Conclusion}

\section{Goal Evaluation}

The goals of this thesis were to design and implement a system that can serve as a framework and foundation to be built upon and expanded for usage in modern applications that need fast communication among a potentially large numbers of devices. 

The application, which is the culmination of this thesis, is a highly modular and extensible framework, capable of running distributed across multiple instances and machines with capability for easy horizontal scaling by way of adding new instances.

In order to demonstrate the functionality of the system, as well as provide an example of the implementation of its highly modular components, a sample application was developed, showcasing the system on representatives of the Web, Mobile, Desktop and Server platforms. The application was also subjugated to multiple forms of testing.

\section{Suggestions for Future Expansion}
It is said that software is like a house, there is never a lack of work and improvements that can be done. This thesis provides a framework that can be further built upon and improved. This Section provides some ideas that were outside the scope of this thesis.

\subsection{Authentication} \label{conc:auth}
The system in its current state already provides a logical division of devices between users and allows for aggregation of users into groups. A possible improvement is evident: add support for user authentication.

\subsection{System Administration}
This improvement suggestion goes hand in hand with the one mentioned above, in Section \ref{conc:auth} \nameref{conc:auth}. Once user authentication is possible, the system can be extended with administration interfaces, allowing users with special privileges to manage devices, users and groups.

\subsection{Encryption}
While at its current state, the system could easily be used to send data with End-to-End encryption by encrypting and decrypting it before sending and after receiving, respectively, optional End-to-End encryption could be built-in in the client libraries.
